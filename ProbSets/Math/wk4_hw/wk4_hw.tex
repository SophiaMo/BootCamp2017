\documentclass[letterpaper,12pt]{article}
\usepackage{array}
\usepackage{threeparttable}
\usepackage{geometry}
\usepackage{mathrsfs}
\geometry{letterpaper,tmargin=1in,bmargin=1in,lmargin=1.25in,rmargin=1.25in}
\usepackage{fancyhdr,lastpage}
\pagestyle{fancy}
\lhead{}
\chead{}
\rhead{}
\lfoot{}
\cfoot{}
\rfoot{\footnotesize\textsl{Page \thepage\ of \pageref{LastPage}}}
\renewcommand\headrulewidth{0pt}
\renewcommand\footrulewidth{0pt}
\usepackage[format=hang,font=normalsize,labelfont=bf]{caption}
\usepackage{listings}
\lstset{frame=single,
  language=Python,
  showstringspaces=false,
  columns=flexible,
  basicstyle={\small\ttfamily},
  numbers=none,
  breaklines=true,
  breakatwhitespace=true
  tabsize=3
}
\usepackage{amsmath}
\usepackage{amssymb}
\usepackage{amsthm}
\usepackage{harvard}
\usepackage{setspace}
\usepackage{float,color}
\usepackage[pdftex]{graphicx}
\usepackage{hyperref}
\hypersetup{colorlinks,linkcolor=red,urlcolor=blue}
\theoremstyle{definition}
\newtheorem{theorem}{Theorem}
\newtheorem{acknowledgement}[theorem]{Acknowledgement}
\newtheorem{algorithm}[theorem]{Algorithm}
\newtheorem{axiom}[theorem]{Axiom}
\newtheorem{case}[theorem]{Case}
\newtheorem{claim}[theorem]{Claim}
\newtheorem{conclusion}[theorem]{Conclusion}
\newtheorem{condition}[theorem]{Condition}
\newtheorem{conjecture}[theorem]{Conjecture}
\newtheorem{corollary}[theorem]{Corollary}
\newtheorem{criterion}[theorem]{Criterion}
\newtheorem{definition}[theorem]{Definition}
\newtheorem{derivation}{Derivation} % Number derivations on their own
\newtheorem{example}[theorem]{Example}
\newtheorem{exercise}[theorem]{Exercise}
\newtheorem{lemma}[theorem]{Lemma}
\newtheorem{notation}[theorem]{Notation}
\newtheorem{problem}[theorem]{Problem}
\newtheorem{proposition}{Proposition} % Number propositions on their own
\newtheorem{remark}[theorem]{Remark}
\newtheorem{solution}[theorem]{Solution}
\newtheorem{summary}[theorem]{Summary}
%\numberwithin{equation}{section}
\bibliographystyle{aer}
\newcommand\ve{\varepsilon}
\newcommand\boldline{\arrayrulewidth{1pt}\hline}


\begin{document}

\begin{flushleft}
  \textbf{\large{Problem Set} 4} \\
  OSM Lab-Math \\
  Sophia Mo
\end{flushleft}

\vspace{5mm}

\noindent\textbf{Problem 1 (6. 1)} \\
The standard form  is 
\begin{align*}
\text{min   }& -e^{-w^Tx}\\
\text{s.t.   }& w^TAw - w^TAy-w^Tx\leq - a\\
& y^Tw - w^Tx = b
\end{align*}

\noindent\textbf{Problem 2 (6. 5)} \\
Denote the quantity for knobs as x, the quantity for milk cartons as y.
\begin{align*}
\text{min   }& -0.05x - 0.07y\\
\text{s.t.   }& 3x + 4y\leq 240,000\\
& x + 2y\leq 100
\end{align*}

\noindent\textbf{Problem 3 (6. 6)} \\
The Jaconbian matrix is\\
\begin{align*}
(6xy+4y^2+y,\text{  }3x^2+8xy+x )
\end{align*}
Setting each entry to zero, we get the critical values are $(x,y) = (0, 0), (0, -\frac{1}{4}), (-\frac{1}{3}, 0), (-\frac{1}{9}, -\frac{1}{12})$.
The Hessian matrix is\\
\begin{align*}
\begin{pmatrix}
6y & 6x+8y+1\\
6x+8y+1 & 8x
\end{pmatrix}
\end{align*}
At $(0, 0)$, the determinant is zero, so the critical value is a saddle point. \\
At $(0, \frac{1}{4})$, $(-\frac{1}{3}, 0)$, and $(-\frac{1}{9}, -\frac{1}{12})$, the determinants are greater than zero, and traces are less than zero. So these three values are local maxima.\\
\\
\noindent\textbf{Problem 4 (6.11)}\\
Suppose the first guess is $x_0$
\begin{align*}
f'(x_0) &= 2ax_0+b\\
f''(x_0) &= 2a
\end{align*}
So the Newton Method gives us $-\frac{b}{2a}$. Plugging this value into the equation, we know that $-\frac{b}{2a}$ is a critical value. Moreover, since the second derivative is always greater than zero, we know the critical value is a minimizer.
\end{document}
