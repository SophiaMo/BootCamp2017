\documentclass[letterpaper,12pt]{article}
\usepackage{array}
\usepackage{threeparttable}
\usepackage{geometry}
\usepackage{mathrsfs}
\geometry{letterpaper,tmargin=1in,bmargin=1in,lmargin=1.25in,rmargin=1.25in}
\usepackage{fancyhdr,lastpage}
\pagestyle{fancy}
\lhead{}
\chead{}
\rhead{}
\lfoot{}
\cfoot{}
\rfoot{\footnotesize\textsl{Page \thepage\ of \pageref{LastPage}}}
\renewcommand\headrulewidth{0pt}
\renewcommand\footrulewidth{0pt}
\usepackage[format=hang,font=normalsize,labelfont=bf]{caption}
\usepackage{listings}
\lstset{frame=single,
  language=Python,
  showstringspaces=false,
  columns=flexible,
  basicstyle={\small\ttfamily},
  numbers=none,
  breaklines=true,
  breakatwhitespace=true
  tabsize=3
}
\usepackage{amsmath}
\usepackage{amssymb}
\usepackage{amsthm}
\usepackage{harvard}
\usepackage{setspace}
\usepackage{float,color}
\usepackage[pdftex]{graphicx}
\usepackage{hyperref}
\hypersetup{colorlinks,linkcolor=red,urlcolor=blue}
\theoremstyle{definition}
\newtheorem{theorem}{Theorem}
\newtheorem{acknowledgement}[theorem]{Acknowledgement}
\newtheorem{algorithm}[theorem]{Algorithm}
\newtheorem{axiom}[theorem]{Axiom}
\newtheorem{case}[theorem]{Case}
\newtheorem{claim}[theorem]{Claim}
\newtheorem{conclusion}[theorem]{Conclusion}
\newtheorem{condition}[theorem]{Condition}
\newtheorem{conjecture}[theorem]{Conjecture}
\newtheorem{corollary}[theorem]{Corollary}
\newtheorem{criterion}[theorem]{Criterion}
\newtheorem{definition}[theorem]{Definition}
\newtheorem{derivation}{Derivation} % Number derivations on their own
\newtheorem{example}[theorem]{Example}
\newtheorem{exercise}[theorem]{Exercise}
\newtheorem{lemma}[theorem]{Lemma}
\newtheorem{notation}[theorem]{Notation}
\newtheorem{problem}[theorem]{Problem}
\newtheorem{proposition}{Proposition} % Number propositions on their own
\newtheorem{remark}[theorem]{Remark}
\newtheorem{solution}[theorem]{Solution}
\newtheorem{summary}[theorem]{Summary}
%\numberwithin{equation}{section}
\bibliographystyle{aer}
\newcommand\ve{\varepsilon}
\newcommand\boldline{\arrayrulewidth{1pt}\hline}


\begin{document}

\begin{flushleft}
  \textbf{\large{Problem Set} 3} \\
  OSM Lab-Math \\
  Sophia Mo
\end{flushleft}

\vspace{5mm}

\noindent\textbf{Problem 1 (4.2)} \\
In Problem 3.38 we have shown that the derivative operator is
\begin{align*}
D =
\begin{pmatrix}
0&1&0\\
0&0&2\\
0&0&0
\end{pmatrix}
\end{align*}
To find the eigenvalue, we can calculate the root to the characteristic polynomial 
\begin{align*}
det(\lambda I - D) = \lambda^3
\end{align*}
Therefore, the only eigenvalue is $\lambda = 0$, with algebraic multiplicity 3 and geometric multiplicity 1. Any vector in the form of $(a, 0, 0), a\in \mathbb{R}$ is an eigenvector and belongs to the eigenspace.\\

\noindent\textbf{Problem 2 (4.4)} \\
(i)\\
Suppose the $A$ is a Hermitian matrix. Let $\lambda$ be any eigenvalue of $A$, with associated eigenvector $v$. Then by definition,
\begin{align*}
Av &= \lambda v\\
(Av)^H &= \bar{\lambda} v^H\\
v^H A^H &= \bar{\lambda} v^H\\
v^HA^Hv &= \bar{\lambda}v^Hv \\
v^H A v &= \bar{\lambda}v^H v\\
v^H \lambda v &=\bar{\lambda} v^H v\\
\lambda v^H v &= \bar{\lambda} v^H v\\
\lambda & = \bar{\lambda}
\end{align*}
For the last equation to hold, $\lambda$ must be real.\\
\\
(ii)\\
Suppose the $A$ is a skew-Hermitian matrix. Let $\lambda$ be any eigenvalue of $A$, with associated eigenvector $v$. Then by definition,
\begin{align*}
Av &= \lambda v\\
(Av)^H &= \bar{\lambda} v^H\\
v^H A^H &= \bar{\lambda} v^H\\
v^HA^Hv &= \bar{\lambda}v^Hv \\
v^H (-A) v &= \bar{\lambda}v^H v\\
v^H (-\lambda) v &=\bar{\lambda} v^H v\\
-\lambda v^H v &= \bar{\lambda} v^H v\\
\lambda & = -\bar{\lambda}
\end{align*}
For the last equation to hold, $\lambda$ must be imaginary.\\

\noindent\textbf{Problem 3 (4.6)} \\
For any $n\times n$ upper triangular matrix A, we know that its eigenvalues are the solution to $p(\lambda) = det(A - \lambda I)$. $\forall a_{i, i}$ on the diagonal of $A$, we know that it is a solutions to $p(\lambda)$ since the determinant of an upper-triangular matrix is the product of all entries on its diagonal. The sum of the algebraic multiplicity of each diagonal entry will sum to $n$, and we know, by proposition 4.1.11, that they are all the eigenvalues $A$ has.\\
\\

\noindent\textbf{Problem 4 (4.8)} \\
(i)\\
Since $V$ is the span of the set $S$, we only need to show that the elements in $S$ are linearly independent. However, in Problem 3.8(i), we have already shown that $S$ is an orthonormal set, so every two elements in $S$ are orthogonal to each other. Linear independence follows, and $S$ is a basis for $V$.\\
\\
(ii)\\
Apply $D$ to each element of the basis, we get:
\begin{align*}
D(sin(x)) &= cos(x)\\
D(cos(x)) &= - sin(x)\\
D(sin(2x))&= 2cos(2x)\\
D(cos(2x))&= -2sin(2x)\\
\end{align*}
The transformations of the basis elements form the columns of the matrix $D$, so 
\begin{align*}
D = \begin{pmatrix}
0 & -1 & 0 & 0\\
1 & 0 & 0 & 0\\
0 & 0 & 0 & -2\\
0 & 0 & 2& 0
\end{pmatrix}
\end{align*}
\\
(iii)\\
From our calculations in (ii) we could see that $\{sin(x), cos(x)\}$ and $\{sin(2x), cos(2x)\}$ are two $D-$invariant subspaces in $V$.\\
\\
\noindent\textbf{Problem 5 (4.13)} \\
We know the columns of $P$ are the right eigenvectors of A. Denote a column by $(x, y)^T$, an eigenvalue by $\lambda$, then\\
\begin{align*}
0.8x + 0.4y &= \lambda x\\
0.2x + 0.6y &= \lambda y
\end{align*}
So $\lambda = 1$ or $0.4$, $(x, y) = (2, 1)$ or $(1, -1)$\\
The transition matrix $P = \begin{pmatrix}
2 & 1\\
1 & -1
\end{pmatrix}$.\\
\\

\noindent\textbf{Problem 6 (4.15)} \\
Let $A$ be a semi-simple matrix. We know that $A$ is diagonalizable, so $A = P^{-1} D P$, for some nonsingular matrix $P$ and diagonal matrix $D$. Thus, 
\begin{align*}
f(A) &= f(P^{-1} D P) = a_0I+a_1P^{-1}DP+...+a_nP^{-1}D^{n}P\\
&= P^{-1} (a_0I+a_1D+...+a_nD^n)P = P^{-1} f(D)P
\end{align*}
Since $\lambda_i$'s, as the eigenvalues of $A$, are the diagonal entries of $D$, from the equation above we see that $f(A)$ is also a diagonalizable matrix, and $f((\lambda_i))$'s, are the diagonal entries of $f(D)$. Therefore, $f(\lambda_i)$'s are eigenvalues of $f(A)$.\\
\\
\noindent\textbf{Problem 7 (4.16)} \\
(i)\\
\begin{align*}
A &= \begin{pmatrix}
2 & 1\\
1 & -1
\end{pmatrix}\begin{pmatrix}
1 & 0\\
0 & 0.4\\
\end{pmatrix}\begin{pmatrix}
\frac{1}{3} & \frac{1}{3}\\
\frac{1}{3} & -\frac{2}{3}
\end{pmatrix}\\
A^k &= \begin{pmatrix}
2 & 1\\
1 & -1
\end{pmatrix}\begin{pmatrix}
1 & 0\\
0 & 0.4^k\\
\end{pmatrix}\begin{pmatrix}
\frac{1}{3} & \frac{1}{3}\\
\frac{1}{3} & -\frac{2}{3}
\end{pmatrix} = \begin{pmatrix}
\frac{1}{3}(2 + 0.4^k) & \frac{1}{3}(2 - 2\times 0.4^k)\\
\frac{1}{3}(1 - 0.4^k) & \frac{1}{3}(1 + 2\times 0.4^k)
\end{pmatrix}\\
\end{align*}
\begin{align*}
lim_{n\rightarrow \infty}A^n = \begin{pmatrix}
\frac{2}{3} & \frac{2}{3}\\
\frac{1}{3} & \frac{1}{3}
\end{pmatrix}
\end{align*}
$||A^k - B||_1 = \frac{4}{3}\times 0.4^k\rightarrow 0$.\\
\\
(ii)\\
$||A^k-B||_{\infty} = 0.4^k \rightarrow 0$\\
$||A^K-B||_F = \frac{\sqrt{10}}{3}\times 0.4^k\rightarrow 0$\\
So the result do not change ($B$ is the same).\\
\\
(iii)\\
By Theorem 4.3.12, the eigenvalues are $3+5+1 = 9$ and $3+5\times 0.4+(0.4)^3 = 5.064$\\
\\
\noindent\textbf{Problem 8 (4.18)} \\
By Proposition 4.1.23, $A$ and $A^T$ have the same characteristic polynomial, so they have the same eigenvalues. Therefore, if $\lambda$ is an eigenvalue of $A$, then it is an eigenvalue of $A^T$, so $\exists$ associated eigenvector $x$ s.t. $A^T x = \lambda x$. Taking the transpose of both sides, we get $x^T A =\lambda x^T$.\\
\\
\noindent\textbf{Problem 9 (4.20)} \\
Suppose $A$ is Hermitian and orthonormally similar to $B$. Then $\exists $ orthonormal matrix $U$ such that $B = U^H A U$. So $B^H = U^H A^H U$. Since $A$ is Hermitian, $A^H = A$, so $B^H = U^H AU =B$. So $B$ is Hermitian.\\
\\
\noindent\textbf{Problem 10 (4.24)} \\
Suppose $A$ is Hermitian, then
\begin{align*}
<x, Ax> = x^HA^H x = x^HAx = <Ax, x> = \overline{<x, Ax>}
\end{align*}
For a number to equal its complex conjugate, the number must be real. Since $||x||^2$ is real, we know that $\rho(x)$ is real.\\
Suppose $A$ is skew-Hermitian, then
\begin{align*}
<x, Ax> = x^HA^H x = x^H(-A)x = <-Ax, x> = -<Ax, x> = -\overline{<x, Ax>}
\end{align*}
For a number to be the opposite of its complex conjugate, the number must be imaginary. Since $||x||^2$ is real, we know that $\rho(x)$ is imaginary.\\
\\
\noindent\textbf{Problem 11 (4.25)} \\
(i)\\
We know $[x_1, ... x_n]$ are orthonormal eigenvectors.\\
$\Rightarrow$ Let $Q = [x_1, x_2, ..., x_n]$ (orthonormal matrix), Problem 3.10 shows that $Q^HQ = QQ^H = I$. But $QQ^H = x_1x_1^H+ ... +x_nx_n^H$, so $x_1x_1^H+ ... +x_nx_n^H = I$.\\
\\
(ii)\\
$\lambda_1x_1x_1^H+...+\lambda_nx_nx_n^H = Ax_1x_1^H+...+Ax_nx_n^H = A(x_1x_1^H +... + x_nx_n^H) = AI = A$\\
\\
\noindent\textbf{Problem 12 (4.27)} \\
Suppose that $A$ is positive definite. Then by Proposition 4.5.7, we know $A$ can be written as $S^HS$, where $S$ is non-singular. Notice that the $i$ the diagonal element of $A$ is then just the square of the $(i, i )$the element of $S$, which must be real and positive.\\
\\
\noindent\textbf{Problem 13 (4.28)} \\
Since $A, B$ are positive semidefinite matrices, we know $A= S^HS, B = T^HT$\\
$\Rightarrow tr(AB) = tr(S^HST^HT) = tr((TS^H)(ST^H)) = tr(U^HU)$, if we let $U = ST^H$.
$\Rightarrow tr(AB)$ is equal to the trace of some semidefinite matrix (by Proposition 4.5.9). Following the proof for Problem 4.27, we know that a semi-definite matrix has non-negative diagonal entries, so its trace $\geq 0$. The first inequality holds.\\
\\
In the space of semi-definite matrices, we can define the inner product by $<A, B> = tr(AB)$. Note that trace is indeed a norm since\\
\begin{align*}
tr(AA) &= \sum(\lambda_i)^2 \geq 0\\
tr(A(bB+cC)) &= btr(AB) + ctr(AC)\\
tr(AB) &= \overline{tr(BA)}\\
\end{align*}
Therefore, by Cauchy-Schwartz Inequality, we know 
\begin{align*}
tr(AB) \leq \sqrt{tr(A^2)tr(B^2)}
\end{align*}
Also, $tr(A^2) = \sum(\lambda_i)^2 \leq (\sum\lambda_i)^2 = [tr(A)]^2.$ The same reasoning applies to $B$.\\
So $\sqrt{tr(A^2)tr(B^2)}\leq tr(A) tr(B)$ (using the fact that the trace of a matrix = sum of its eigenvalues), and the second inequality holds.\\
\\
\noindent\textbf{Problem 14 (4.31)} \\
(i)\\
$||A||_2 = sup_{x\neq 0}\frac{||Ax||_2}{||x||_2} = sup_{||x||=1}||Ax||_2$\\
So we want to 
\begin{align*}
max\text{  }f = x^HA^HAX \text{ s.t. } g = x^Hx = 1
\end{align*}
Introducing the lagrange multiplier $\lambda$, we know the solutions occur when $\frac{\partial f}{\partial x} - \lambda\frac{\partial g}{\partial x} = 0$.\\
$\Rightarrow (A^HA-\lambda I)x = 0\Rightarrow (A^HA- \lambda I) = 0 (*)$ since $x \neq 0$.\\
$\Rightarrow x^HA^HAx = x^H\lambda I x = \lambda x^H x = \lambda$\\
So we want to find the maximal value that satisfies equation $(*)$, which is equivalent to finding the largest eigenvalue of $A^HA.$\\
It then follow that $||A||_2 = \sigma_1$, the largest singular value of $A$.\\
\\
(ii)\\
Using SVD, $A$ could be decomposed into $U\Sigma V^H$, where $U, V$ are orthonormal matrices. So $A^{-1} = (U\Sigma V^H)^{-1} = V\Sigma^{-1}U^{-1}$.\\
By $(i)$, we know that $||A^{-1}||_2$ is equal to the maximal diagonal entry of $\Sigma^{-1}$, which is equal to $\sigma_n^{-1}$, where $\sigma_n$ is the smallest singular value of $A$.\\
\\
(iii)\\
Let $\lambda$ be an eigenvalue of $A^HA$, then $\exists x$ s.t. $A^HAx = \lambda x \Rightarrow AA^H(Ax) = A\lambda x = \lambda (Ax)$. So we know $\lambda$ is an eigenvalue of $AA^H$. Conversely, let $\lambda$ be an eigenvalue of $AA^H$, then $AA^Hx = \lambda x\Rightarrow A^HA(A^Hx) = \lambda (A^Hx)$, so $\lambda$ is an eigenvalue of $AA^H$.\\
Therefore, we know $A^HA$ and $AA^H$ have the same eigenvalues, which indicates that $A^H$ and $A$ have the same singular values. By $(i)$, we then know that $||A||_2 = ||A^H||_2$.\\
\\
Since $(A^T)^H A^T = AA^H$, $A^T$ and $A^H$ have the singular eigenvalues, so $||A^H||_2 = ||A^T||_2$\\
\\
$(A^HA)^H(A^HA) = A^HAA^HA$. Let $\lambda$ be a singular value for $A^HA$, then $\exists x$ s.t. $(A^HA)^H(A^HA) x = \lambda^2 x\Rightarrow (A^H A)(A^H A x) = \lambda(\lambda x)$, so $\lambda$ is an eigenvalue for $A^HA$. Conversely, let $\lambda$ be an eigenvalue for $A^HA$, then $(A^HA)x = \lambda x\Rightarrow (A^HA)^H(A^HA) x = \lambda^2 x$, so $\lambda$ is a singular value of $A^HA$. Thus the eigenvalues of $A^HA$ are the same as its singular values. Since $||A^HA||_2$ is its maximal singular value, it is also its maximal eigenvalue, which is equal to $||A||_2^2$\\
\\
(iv)\\
As proved in Problem 3.29, orthonormal matrices have norm 1. Moreover, we know $\forall \text{ matrix }A,B$, $||AB||_2\leq ||A||_2||B||_2$.\\
So $||UAV||_2 \leq ||U||_2||A||_2||V||_2 = ||A||_2$.\\
As proved in 3.10, the inverse of an orthonormal matrix is also orthonormal. So $||A||_2 \leq ||U^{-1}UAVV^{-1}||_2 = ||U^{-1}||_2 ||UAV||_2 ||V^{-1}||_2 = ||UAV||_2$\\
$\Rightarrow ||UAV||_2 = ||A||_2$\\
\\
\noindent\textbf{Problem 15 (4.32)} \\
(i)\\
$||UAV||_F = \sqrt{tr[(UAV)^H(UAV)} = \sqrt{tr(V^HA^HU^HUAV)} = \sqrt{V^HA^HAV} = \sqrt{A^HAV^HV} = \sqrt{A^HA} =||A||_F$\\
\\
(ii)\\
$||A||_F = \sqrt{tr(A^HA)} = \sqrt{\sum{\lambda_i}}$, where $\lambda_i$'s are the eigenvalues of $A^HA$. \\
$\Rightarrow ||A||_F = (\sigma_1+...+\sigma_r^2)^{\frac{1}{2}}$.\\
\\
\noindent\textbf{Problem 16 (4.33)} \\
We know $||A||_2 = \sigma_1$, the maximal diagonal element of $\Sigma$, and that $A = U\Sigma V^H\Rightarrow U^HAV = \Sigma$, where $U, U^H, V, V^H$ are all orthonormal matrices.\\
Note that each column of an orthonormal matrix has norm 1, so the diagonal elements of $\Sigma$ could be written as $y^HAx$, for some $y, x$ with norm one.\\
Finding the maximal $\sigma$ is therefore equivalent to finding $sup_{||x||_2=||y||_2=1}y^HAx$.\\
\\
\noindent\textbf{Problem 17 (4.36)} \\
For example, $\begin{pmatrix}
1 & 2\\
0 & 3
\end{pmatrix}$
has eigenvalues 1 and 3, but its singular value $2\sqrt{10} + 7$, is not equal to any of the eigenvalues.\\
\\
\noindent\textbf{Problem 18 (4.38)} \\
(i)\\
$AA^{\dag}A = U_1\Sigma_1V_1^H V_1\Sigma_1^{-1}U_1^H U_1\Sigma_1V_1^H  =U_1\Sigma_1\Sigma_1^{-1}\Sigma_1V_1^H = A$\\
\\
(ii)\\
$A^{\dag}AA^{\dag} = V_1\Sigma_1^{-1}U_1^H U_1\Sigma_1V_1^H V_1\Sigma_1^{-1}U_1^H = V_1 \Sigma_1^{-1} \Sigma \Sigma_1^{-1} U_1^H = V_1\Sigma_1^{-1}U_1^H = A^{\dag}$\\
\\
(iii)\\
$(AA^{\dag})^H = (U_1\Sigma_1V_1^H V_1\Sigma_1^{-1}U_1^H)^H = U_1\Sigma_1V_1^H V_1\Sigma_1^{-1}U_1^H = AA^{\dag} $\\
\\
(iv)\\
$(A^{\dag}A)^H = (V_1\Sigma_1^{-1}U_1^H U_1\Sigma_1V_1^H)^H = V_1\Sigma_1^{-1}U_1^H U_1\Sigma_1V_1^H = A^{\dag} A$\\
\\
(v)\\
Denote the transformation $AA^{\dag}$ by $P$. Since $P^2 = AA^{\dag} AA^{\dag} =  AA^{\dag} = P$, by Problem 3.47 we know that $P$ is a projection. Also notice that $PA = AA^{\dag}A = A$.\\
$\forall y\in \mathscr{R}(A), \exists x$ s.t. $Ax = y \Rightarrow Ax = PAx = Py = y$ .\\
If $Py = y$, then $y = AA^{\dag} y$, clearly $y\in \mathscr{R}(A)$.\\
So $P = AA^{\dag} = proj_{\mathscr{R}(A)}$\\
\\
(vi)\\
Following the previous reasoning, we first notice that $Q^2 = Q$, so $Q$ is a projection matrix, and $QA^{\dag} = A^{\dag}$.\\
$\forall y\in \mathscr{R}(A^H), \exists x$ s.t. $A^Hx = y \Rightarrow A^Hx = QA^Hx = Qy = y$ .\\
If $Qy = y$, then $y = A^{\dag}A y$, clearly $y\in \mathscr{R}(A^H)$.\\
So $Q = A^{\dag}A = proj_{\mathscr{R}(A^H)}$\\
\end{document}